\documentclass{article}
\usepackage[utf8]{inputenc}
\usepackage[margin=1in,headheight=24pt]{geometry}
\usepackage{amsmath}
\usepackage{lastpage}
\usepackage{fancyhdr}
\pagestyle{fancy}
\allowdisplaybreaks

\fancyhead{}
\fancyhead[L]{EGEE-2110 \\ Engineering Analysis}
\fancyhead[C]{Week 3 Homework}
\fancyhead[R]{02/05/2021 \\ Kaitlyn Wiseman}
\renewcommand{\footrulewidth}{0.4pt}
\fancyfoot{}
\fancyfoot[R]{\thepage\ of \pageref{LastPage}}


\begin{document}

{\large \noindent Problem Set:}

\par 7.4 [2, 4, 18, 20]
\par 7.7 [12, 14]
\par 7.8 [4, 6, 8]
\vspace{5mm}

\noindent \hrulefill

\section*{7.4}
\setcounter{equation}{0}

\subsection*{7.4 [2]}

Find the rank.  Find a basis for the row space.  Find a basis for the column space.  \textit{Hint.}  Row-reduce the matrix and its transpose.  Row-reduce the matrix and its transpose.  (You may omit obvious factors from the vectors of these bases.) 

\begin{center}
    $\begin{bmatrix} 
    a & b\\
    b & a
    \end{bmatrix}$
\end{center}

Let

\begin{align}
    \label{eq:1}
    && \textbf{A} &= \begin{bmatrix} 
    a & b\\
    b & a
    \end{bmatrix}
    \\
    \label{eq:2}
    \textbf{R\textsubscript{2}} = \textbf{R\textsubscript{2}} + (-\frac{b}{a})\textbf{R\textsubscript{1}} \longrightarrow && &= \begin{bmatrix}
    a & b\\
    b - \frac{b}{a}(a) & a - \frac{b}{a}(b)
    \end{bmatrix}
    \\
    \label{eq:3}
    && &= \begin{bmatrix}
    a & b\\
    0 & \frac{a^2 - b^2}{a}
    \end{bmatrix}
    \\
    \label{eq:4}
    \eqref{eq:1}
    && \textbf{A\textsuperscript{T}} &= \begin{bmatrix} 
    a & b\\
    b & a
    \end{bmatrix}
    \\
    \label{eq:5}
    \eqref{eq:1}
    && &= \textbf{A}
    \\
    \label{eq:6}
    \eqref{eq:3} && &= \begin{bmatrix}
    a & b\\
    0 & \frac{a^2 - b^2}{a}
    \end{bmatrix}
    \\
    \label{eq:7}
    \eqref{eq:3}, \eqref{eq:6} &&
    \text{rank \textbf{A}} &= 2
    \\
    \label{eq:8}
    \eqref{eq:3}, \eqref{eq:6} && \text{Both the row and column bases } &\text{are}
    \{ \begin{bmatrix} a & b \end{bmatrix}, \begin{bmatrix}
    0 & \frac{b^2 - a^2}{b}
    \end{bmatrix} \}
\end{align} 

\subsection*{7.4 [4]}
\setcounter{equation}{0}

Find the rank.  Find a basis for the row space.  Find a basis for the column space.  \textit{Hint.}  Row-reduce the matrix and its transpose.  (You may omit obvious factors from the vectors of these bases.)

Let

\begin{align}
    \label{eq:1}
    && \textbf{B} &= \begin{bmatrix}
    6 & -4 & 0\\
    -4 & 0 & 2\\
    0 & 2 & 6
    \end{bmatrix}
    \\
    \label{eq:2}
    \textbf{R\textsubscript{2}} = \textbf{R\textsubscript{2}} + \frac{2}{3}\textbf{R\textsubscript{1}} \longrightarrow && &= \begin{bmatrix}
    6 & -4 & 0\\
    -4 + \frac{2}{3}{6} & 0 - \frac{2}{3}(4) & 2\\
    0 & 2 & 6
    \end{bmatrix}
    \\
    \label{eq:3}
    && &= \begin{bmatrix}
    6 & -4 & 0\\
    0 & -\frac{8}{3} & 2\\
    0 & 2 & 6
    \end{bmatrix}
    \\
    \label{eq:4}
    \textbf{R\textsubscript{3}} = \textbf{R\textsubscript{3}} + \frac{3}{4}\textbf{R\textsubscript{2}} \longrightarrow && &= \begin{bmatrix}
    6 & -4 & 0\\
    0 & -\frac{8}{3} & 2\\
    0 & 2+(\frac{3}{4})(-\frac{8}{3}) & 6 + \frac{3}{4}(2)
    \end{bmatrix}
    \\
    \label{eq:5}
    && &= \begin{bmatrix}
    6 & -4 & 0\\
    0 & -\frac{8}{3} & 2\\
    0 & 0 & \frac{15}{2}
    \end{bmatrix}
    \\
    \label{eq:6}
    \eqref{eq:1} && \textbf{B\textsuperscript{T}} &= \begin{bmatrix}
    6 & -4 & 0\\
    -4 & 0 & 2\\
    0 & 2 & 6
    \end{bmatrix}
    \\
    \label{eq:7}
    \eqref{eq:1} && &= \textbf{B}
    \\
    \label{eq:8}
    \eqref{eq:5}, \eqref{eq:7} && \text{rank \textbf{A}} &= 3
    \\
    \label{eq:9}
    \eqref{eq:5} && \text{A row-space basis} &\text{: } \{ \begin{bmatrix}6&-4&0\end{bmatrix}, \begin{bmatrix}0&-\frac{8}{3}&2\end{bmatrix}, \begin{bmatrix}0&0&\frac{15}{2}\end{bmatrix}\}
    \\
    \label{eq:10}
    \eqref{eq:5}, \eqref{eq:7} && \text{A column-space basis} &\text{: } \{
    \begin{bmatrix}6\\-4\\0\end{bmatrix}, \begin{bmatrix}
    0\\-\frac{8}{3}\\2
    \end{bmatrix}, \begin{bmatrix}
    0\\0\\\frac{15}{2}
    \end{bmatrix}\}
\end{align}

\subsection*{7.4 [18]}
\setcounter{equation}{0}

Are the following sets of vectors linearly independent?  Show the details of your work.

$ \begin{bmatrix}1&\frac{1}{2}&\frac{1}{3}&\frac{1}{4}\end{bmatrix}, \begin{bmatrix}
\frac{1}{2}&\frac{1}{3}&\frac{1}{4}&\frac{1}{5}
\end{bmatrix}, \begin{bmatrix}
\frac{1}{3}&\frac{1}{4}&\frac{1}{5}&\frac{1}{6}
\end{bmatrix}, \begin{bmatrix}
\frac{1}{4}&\frac{1}{5}&\frac{1}{6}&\frac{1}{7}
\end{bmatrix}$

Make a matrix:

\begin{align}
    \label{eq:1}
    && &\text{ } \begin{bmatrix}
    1&\frac{1}{2}&\frac{1}{3}&\frac{1}{4}\\
    \frac{1}{2}&\frac{1}{3}&\frac{1}{4}&\frac{1}{5}\\
    \frac{1}{3}&\frac{1}{4}&\frac{1}{5}&\frac{1}{6}\\
    \frac{1}{4}&\frac{1}{5}&\frac{1}{6}&\frac{1}{7}
    \end{bmatrix}
    \\
    \label{eq:2}
    \textbf{R\textsubscript{2}} = \textbf{R\textsubscript{2}} - \frac{1}{2}\textbf{R\textsubscript{1}}\longrightarrow && &= \begin{bmatrix}
    1&\frac{1}{2}&\frac{1}{3}&\frac{1}{4}\\
    0&\frac{1}{12}&\frac{1}{12}&\frac{3}{40}\\
    \frac{1}{3}&\frac{1}{4}&\frac{1}{5}&\frac{1}{6}\\
    \frac{1}{4}&\frac{1}{5}&\frac{1}{6}&\frac{1}{7}
    \end{bmatrix}
    \\
    \label{eq:3}
    \textbf{R\textsubscript{3}} = \textbf{R\textsubscript{3}} - \frac{1}{3}\textbf{R\textsubscript{1}}, \textbf{R\textsubscript{4}} = \textbf{R\textsubscript{4}} - \frac{1}{4}\textbf{R\textsubscript{1}} \longrightarrow && &= \begin{bmatrix}
    1&\frac{1}{2}&\frac{1}{3}&\frac{1}{4}\\
    0&\frac{1}{12}&\frac{1}{12}&\frac{3}{40}\\
    0&\frac{1}{12}&\frac{4}{45}&\frac{1}{12}\\
    0&\frac{1}{30}&\frac{1}{12}&\frac{9}{112}
    \end{bmatrix}
    \\
    \label{eq:4}
    \textbf{R\textsubscript{3}} = \textbf{R\textsubscript{3}} - \textbf{R\textsubscript{2}}, \textbf{R\textsubscript{4}} = \textbf{R\textsubscript{4}} - \frac{2}{5}\textbf{R\textsubscript{2}} \longrightarrow && &= \begin{bmatrix}
    1&\frac{1}{2}&\frac{1}{3}&\frac{1}{4}\\
    0&\frac{1}{12}&\frac{1}{12}&\frac{3}{40}\\
    0&0&\frac{1}{180}&\frac{1}{120}\\
    0&0&\frac{1}{20}&\frac{141}{2800}
    \end{bmatrix}
    \\
    \label{eq:5}
    \textbf{R\textsubscript{4}} = \textbf{R\textsubscript{4}} - 9\textbf{R\textsubscript{3}} \longrightarrow && &= \begin{bmatrix}
    1&\frac{1}{2}&\frac{1}{3}&\frac{1}{4}\\
    0&\frac{1}{12}&\frac{1}{12}&\frac{3}{40}\\
    0&0&\frac{1}{180}&\frac{1}{120}\\
    0&0&0&-\frac{69}{2800}
    \end{bmatrix}
    \\
    \label{eq:6}
    &&
    \text{rank } &= 4
\end{align}

The number of vectors and the rank of the matrix they form are both 4.  Hence, these vectors are \textbf{linearly independent}.

\subsection*{7.4 [20]}
\setcounter{equation}{0}

Are the following sets of vectors linearly independent?  Show the details of your work.

$
\begin{bmatrix}
1&2&3&4
\end{bmatrix}, \begin{bmatrix}
2&3&4&5
\end{bmatrix}, \begin{bmatrix}
3&4&5&6
\end{bmatrix}, \begin{bmatrix}
4&5&6&7
\end{bmatrix}
$

Make a matrix:

\begin{align}
    \label{eq:1}
    && &\text{ } \begin{bmatrix}
    1&2&3&4\\
    2&3&4&5\\
    3&4&5&6\\
    4&5&6&7
    \end{bmatrix}
    \\
    \label{eq:2}
    \begin{pmatrix}
    \textbf{R\textsubscript{2}} = \textbf{R\textsubscript{2}} - 2\textbf{R\textsubscript{1}}\\
    \textbf{R\textsubscript{3}} = \textbf{R\textsubscript{3}} - 3\textbf{R\textsubscript{1}}\\
    \textbf{R\textsubscript{4}} = \textbf{R\textsubscript{4}} - 4\textbf{R\textsubscript{1}}
    \end{pmatrix} \longrightarrow && &=\begin{bmatrix}
    1&2&3&4\\
    0&-1&-2&-3\\
    0&-2&-4&-6\\
    0&-3&-6&-9
    \end{bmatrix}
    \\
    \label{eq:3}
    \begin{pmatrix}
    \textbf{R\textsubscript{3}} = \textbf{R\textsubscript{3}} - 2\textbf{R\textsubscript{2}}\\
    \textbf{R\textsubscript{4}} = \textbf{R\textsubscript{4}} - 3\textbf{R\textsubscript{2}} 
    \end{pmatrix} \longrightarrow && &= \begin{bmatrix}
    1&2&3&4\\
    0&-1&-2&-3\\
    0&0&0&0\\
    0&0&0&0
    \end{bmatrix}
    \\
    \label{eq:4}
    \eqref{eq:3} && \text{rank} &= 2
\end{align}

The number of vectors is 4, and the rank of the matrix they form is 2.  Hence, these vectors are \textbf{linearly dependent}.

\newpage

\section*{7.7}
\setcounter{equation}{0}

\subsection*{7.7 [12]}
\setcounter{equation}{0}

Showing the details, evaluate:

\begin{align}
    \label{eq:1}
    && \begin{vmatrix}
    a & b & c\\
    c & a & b\\
    b & c & a
    \end{vmatrix} &= a\begin{vmatrix} a & b\\c & a \end{vmatrix} - b\begin{vmatrix}c&b\\b&a\end{vmatrix} + c\begin{vmatrix}c&a\\b&c\end{vmatrix}
    \\
    \label{eq:2}
    && &= a(a*a-b*c) - b(c*a-b*b) + c(c*c-a*b)
    \\
    \label{eq:3}
    && &= a^3-abc-abc+b^3+c^3-abc
    \\
    \label{eq:4}
    && &= a^3+b^3+c^3-3abc
\end{align}

\subsection*{7.7 [14]}
\setcounter{equation}{0}

Showing the details, evaluate:

\begin{align}
    \label{eq:1}
    && \begin{vmatrix}
    4&7&0&0\\
    2&8&0&0\\
    0&0&1&5\\
    0&0&-2&2 \end{vmatrix} &= 4\begin{vmatrix}8&0&0\\0&1&5\\0&-2&2\end{vmatrix}-7\begin{vmatrix}2&0&0\\0&1&5\\0&-2&2\end{vmatrix} + 0+0
    \\
    \label{eq:2}
    && &= 4 \begin{pmatrix} 8\begin{vmatrix}1&5\\-2&2 \end{vmatrix} +0+0\end{pmatrix}-7\begin{pmatrix}
    2\begin{vmatrix}1&5\\-2&2\end{vmatrix}+0+0
    \end{pmatrix}
    \\
    \label{eq:3}
    && &= 32(1*2+2*5)-14(1*2+2*5)
    \\
    \label{eq:4}
    && &= 216
\end{align}

\newpage

\section*{7.8}
\setcounter{equation}{0}

\subsection*{7.8 [4]}
\setcounter{equation}{0}

Find the inverse by Gauss-Jordan. Check.

\begin{align}
    \label{eq:1}
    && \textbf{A} &= \begin{bmatrix}
    0&0&0.1\\
    0&-0.4&0\\
    2.5&0&0
    \end{bmatrix}
    \\
    \label{eq:2}
    \text{Make aug. mx:} && &\text{  } \begin{bmatrix}
    0 & 0 & 0.1 &|& 1 & 0 & 0\\
    0 & -0.4 & 0 &|& 0 & 1 & 0\\
    2.5 & 0 & 0 &|& 0 & 0 & 1
    \end{bmatrix}
    \\
    \label{eq:3}
    \text{switch \textbf{R\textsubscript{1}}, \textbf{R\textsubscript{3}}} \longrightarrow && &\text{ } \begin{bmatrix}
    2.5 & 0 & 0 &|& 0 & 0 & 1\\
    0 & -0.4 & 0 &|& 0 & 1 & 0\\
    0 & 0 & 0.1 &|& 1 & 0 & 0
    \end{bmatrix}
    \\
    \label{eq:4}
    \textbf{R\textsubscript{1}} = \frac{1}{2.5}\textbf{R\textsubscript{1}} \longrightarrow && &\text{ } \begin{bmatrix}
    1 & 0 & 0 &|& 0 & 0 & 0.4\\
    0 & -0.4 & 0 &|& 0 & 1 & 0\\
    0 & 0 & 0.1 &|& 1 & 0 & 0
    \end{bmatrix}
    \\
    \label{eq:5}
    \textbf{R\textsubscript{2}} = -\frac{1}{0.4}\textbf{R\textsubscript{2}} \longrightarrow && &\text{ } \begin{bmatrix}
    1 & 0 & 0 &|& 0 & 0 & 0.4\\
    0 & 1 & 0 &|& 0 & -2.5 & 0\\
    0 & 0 & 0.1 &|& 1 & 0 & 0
    \end{bmatrix}
    \\
    \label{eq:6}
    \textbf{R\textsubscript{3}} = \frac{1}{0.1}\textbf{R\textsubscript{3}} \longrightarrow && &\text{ } \begin{bmatrix}
    1 & 0 & 0 &|& 0 & 0 & 0.4\\
    0 & 1 & 0 &|& 0 & -2.5 & 0\\
    0 & 0 & 1 &|& 10 & 0 & 0
    \end{bmatrix}
    \\
    \label{eq:7}
    \eqref{eq:6} && \textbf{A\textsuperscript{-1}} &= \begin{bmatrix}
    0 & 0 & 0.4\\
    0 & -2.5 & 0\\
    10 & 0 & 0
    \end{bmatrix}
    \\
    \label{eq:8}
    \text{Check:} && \textbf{AA\textsuperscript{-1}} &= \begin{bmatrix}
    0&0&0.1\\
    0&-0.4&0\\
    2.5&0&0
    \end{bmatrix}\begin{bmatrix}
    0 & 0 & 0.4\\
    0 & -2.5 & 0\\
    10 & 0 & 0
    \end{bmatrix}
    \\
    \label{eq:9}
    && &= \begin{bmatrix}
    1&0&0\\
    0&1&0\\
    0&0&1
    \end{bmatrix}
    \\
    \label{eq:10}
    && &= \textbf{I}
\end{align}

\subsection*{7.8 [6]}
\setcounter{equation}{0}

Find the inverse by Gauss-Jordan. Check.

\begin{align}
    \label{eq:1}
    && \textbf{A} &= \begin{bmatrix}
    -4 & 0 & 0\\
    0 & 8 & 13\\
    0 & 3 & 5
    \end{bmatrix}
    \\
    \label{eq:2}
    \text{Make aug. mx.:} && &\text{ } \begin{bmatrix}
    -4 & 0 & 0 &|& 1 & 0 & 0\\
    0 & 8 & 13 &|& 0 & 1 & 0\\
    0 & 3 & 5 &|& 0 & 0 & 1
    \end{bmatrix}
    \\
    \label{eq:3}
    \textbf{R\textsubscript{2}} = \textbf{R\textsubscript{2}} - \frac{13}{5}\textbf{R\textsubscript{3}} \longrightarrow && &\text{ } \begin{bmatrix}
    -4 & 0 & 0 &|& 1 & 0 & 0\\
    0 & \frac{1}{5} & 0 &|& 0 & 1 & -\frac{13}{5}\\
    0 & 3 & 5 &|& 0 & 0 & 1
    \end{bmatrix}
    \\
    \label{eq:4}
    \textbf{R\textsubscript{3}} = \textbf{R\textsubscript{3}} -15\textbf{R\textsubscript{2}} \longrightarrow && &\text{ } \begin{bmatrix}
    -4 & 0 & 0 &|& 1 & 0 & 0\\
    0 & \frac{1}{5} & 0 &|& 0 & 1 & -\frac{13}{5}\\
    0 & 0 & 5 &|& 0 & -15 & 40
    \end{bmatrix}
    \\
    \label{eq:5}
    \textbf{R\textsubscript{1}} = -\frac{1}{4}\textbf{R\textsubscript{1}} \longrightarrow && &\text{ } \begin{bmatrix}
    1 & 0 & 0 &|& -\frac{1}{4} & 0 & 0\\
    0 & \frac{1}{5} & 0 &|& 0 & 1 & -\frac{13}{5}\\
    0 & 0 & 5 &|& 0 & -15 & 40
    \end{bmatrix}
    \\
    \label{eq:6}
    \textbf{R\textsubscript{2}} = 5\textbf{R\textsubscript{2}} \longrightarrow && &\text{ } \begin{bmatrix}
    1 & 0 & 0 &|& -\frac{1}{4} & 0 & 0\\
    0 & 1 & 0 &|& 0 & 5 & -13\\
    0 & 0 & 5 &|& 0 & -15 & 40
    \end{bmatrix}
    \\
    \label{eq:7}
    \textbf{R\textsubscript{3}} = \frac{1}{5}\textbf{R\textsubscript{3}} \longrightarrow && &\text{ } \begin{bmatrix}
    1 & 0 & 0 &|& -\frac{1}{4} & 0 & 0\\
    0 & 1 & 0 &|& 0 & 5 & -13\\
    0 & 0 & 1 &|& 0 & -3 & 8
    \end{bmatrix}
    \\
    \label{eq:8}
    \eqref{eq:7} && \textbf{A\textsuperscript{-1}} &= \begin{bmatrix}
    -\frac{1}{4} & 0 & 0\\
    0 & 5 & -13\\
    0 & -3 & 8
    \end{bmatrix}
    \\
    \label{eq:9}
    \text{Check:} && \textbf{AA\textsuperscript{-1}} &= \begin{bmatrix}
    -4 & 0 & 0\\
    0 & 8 & 13\\
    0 & 3 & 5
    \end{bmatrix}\begin{bmatrix}
    -\frac{1}{4} & 0 & 0\\
    0 & 5 & -13\\
    0 & -3 & 8
    \end{bmatrix}
    \\
    \label{eq:10}
    && &=\begin{bmatrix}
    1&0&0\\
    0&1&0\\
    0&0&1
    \end{bmatrix}
    \\
    \label{eq:11}
    && &= \textbf{I}
\end{align}

\subsection*{7.8 [8]}
\setcounter{equation}{0}

Find the inverse by Gauss-Jordan. Check.

\begin{align}
    \label{eq:1}
    && \textbf{A} &= \begin{bmatrix}
    1 & 2 & 3\\
    4 & 5 & 6\\
    7 & 8 & 9
    \end{bmatrix}
    \\
    \label{eq:2}
    && \text{det \textbf{A}} &= \begin{vmatrix}
    1 & 2 & 3\\
    4 & 5 & 6\\
    7 & 8 & 9
    \end{vmatrix}
    \\
    \label{eq:3}
    && &= 1\begin{vmatrix}
    5 & 6\\ 8 & 9
    \end{vmatrix} -2 \begin{vmatrix}
    4 & 6\\ 7 & 9
    \end{vmatrix} + 3\begin{vmatrix}
    4 & 5\\7&8\end{vmatrix}
    \\
    \label{eq:4}
    && &= (5*9-6*8)-2(4*9-7*6)+3(4*8-5*7)
    \\
    \label{eq:5}
    && &= -3 +12 -9
    \\
    \label{eq:6}
    && &= 0
\end{align}

Hence, the inverse of \textbf{A} does not exist.

\end{document}