\documentclass{article}
\usepackage[utf8]{inputenc}
\usepackage[margin=1in,headheight=24pt]{geometry}
\usepackage{amsmath}
\usepackage{empheq}
\usepackage{lastpage}
\usepackage{fancyhdr}
\pagestyle{fancy}
\allowdisplaybreaks

\fancyhead{}
\fancyhead[L]{EGEE-2110 \\ Engineering Analysis}
\fancyhead[C]{Week 2 Homework}
\fancyhead[R]{01-29-2021 \\ Kaitlyn Wiseman}
\renewcommand{\footrulewidth}{0.4pt}
\fancyfoot{}
\fancyfoot[R]{\thepage\ of \pageref{LastPage}}


\begin{document}

{\large \noindent Problem Set:}

\par 7.1 [6, 8, 14]
\par 7.2 [6, 12, 16, 18]
\par 7.3 [2, 8]
\vspace{5mm}

\noindent \hrulefill

\section*{7.1}
\setcounter{equation}{0}

\subsection*{7.1 [6]}

\par If a $12\times12$ matrix {\textbf{A}} shows the distances between 12 cities in kilometers, how can you obtain from \textbf{A} the matrix \textbf{B} showing these distances in miles?

\begin{center}
    1 mi $\approx$ 1.609 km
\end{center}
\begin{equation}
    \textbf{B} \approx \frac{1}{1.609} \textbf{A}
\end{equation}
\begin{equation}
    \textbf{B} \approx 0.6214\textbf{A}
\end{equation}

\subsection*{7.1 [8]}
\setcounter{equation}{0}
\par Find the following expressions, indicating which of the rules in (3) or (4) they illustrate, or give reasons why they are not defined.

\def \Amatrix {
    \begin{bmatrix}
    0 & 2 & 4\\
    6 & 5 & 5\\
    1 & 0 & -3
    \end{bmatrix}
}
\def \Bmatrix {
    \begin{bmatrix}
    0 & 5 & 2\\
    5 & 3 & 4\\
    -2 & 4 & -2
    \end{bmatrix}
}

\begin{align} 
    \label{eq:3} 
    \textbf{A} &= \Amatrix
    \\
    \label{eq:4}
    \textbf{B} &= \Bmatrix 
\end{align}

\subsubsection*{i) 2\textbf{A} + 4\textbf{B}} 

\begin{align}
    \label{eq:5} 
    \eqref{eq:3}, \eqref{eq:4} &&
    2\textbf{A} + 4\textbf{B} &= 2\Amatrix + 4\Bmatrix
    \\
    \label{eq:6} 
    &&
    &=
    \begin{bmatrix}
    0 & 4 & 8\\
    12 & 10 & 10\\
    2 & 0 & -6
    \end{bmatrix}
    +
    \begin{bmatrix}
    0 & 20 & 8\\
    20 & 12 & 16\\
    -8 & 16 & -8
    \end{bmatrix}
    \\
    \label{eq:7}
    &&
    &=
    \begin{bmatrix}
    0 & 24 & 16\\
    32 & 22 & 26\\
    -6 & 16 & -14
    \end{bmatrix}
\end{align}

\subsubsection*{ii) 4\textbf{B} + 2\textbf{A}}

\begin{align}
    \label{eq:8}
    \text{by 3a} && 4\textbf{B}+2\textbf{A}&=2\textbf{A}+4\textbf{B}
    \\
    \label{eq:9}
    \eqref{eq:7} && &=
    \begin{bmatrix}
    0 & 24 & 16\\
    32 & 22 & 26\\
    -6 & 16 & -14
    \end{bmatrix}
\end{align}

\subsubsection*{iii) 0\textbf{A} + \textbf{B}}

\begin{align}
    \label{eq:10} 
    && 0\textbf{A} + \textbf{B} &= \textbf{0} + \textbf{B}
    \\
    \label{eq:11}
    \text{by 3c} && &= \textbf{B}
    \\
    \label{eq:12}
    \eqref{eq:4} && &= \Bmatrix
\end{align}

\subsubsection*{iv) 0.4\textbf{B} - 4.2\textbf{A}}

\begin{align}
    \label{eq:13}
    \eqref{eq:3}, \eqref{eq:4} && 0.4\textbf{B} - 4.2\textbf{A} &= 0.4\Bmatrix -4.2\Amatrix
    \\
    \label{eq:14}
    && &= 
    \begin{bmatrix}
    0 & 2 & 0.8\\
    2 & 1.2 & 1.6\\
    -0.8 & 1.6 & -0.8
    \end{bmatrix}
    +
    \begin{bmatrix}
    0 & -8.4 & -16.8\\
    -25.2 & -21 & -21\\
    -4.2 & 0 & 12.6
    \end{bmatrix}
    \\
    \label{eq:15}
    && &=
    \begin{bmatrix}
    0 & -6.4 & -16\\
    -23.2 & -19.8 & -19.4\\
    -5 & 1.6 & 11.8
    \end{bmatrix}
\end{align}

\subsection*{7.1 [14]}
\setcounter{equation}{0}

\par Find the following expressions, indicating which of the rules in (3) or (4) they illustrate, or give reasons why they are not defined.

\def \umatrix {
\begin{bmatrix}
1.5\\
0\\
-3.0
\end{bmatrix}
}
\def \vmatrix {
\begin{bmatrix}
-1\\
3\\
2
\end{bmatrix}
}
\def \wmatrix {
\begin{bmatrix}
-5\\
-30\\
10
\end{bmatrix}
}
\def \Ematrix {
\begin{bmatrix}
0 & 2\\
3 & 4\\
3 & -1
\end{bmatrix}
}

\begin{align}
    \label{eq:16}
    \textbf{u} = \umatrix && \textbf{v} = \vmatrix && \textbf{w} = \wmatrix && \textbf{E} = \Ematrix{}
\end{align}

\subsubsection*{i) (5\textbf{u} + 5\textbf{v}) - $\frac{1}{2}$\textbf{w}}

\begin{align}
    \label{eq:17}
    \text{by 4a} && (5\textbf{u} + 5\textbf{v}) - \frac{1}{2}\textbf{w} &= 5(\textbf{u} + \textbf{v}) - \frac{1}{2}\textbf{w}
    \\
    \label{eq:18}
    \eqref{eq:16} && &= 5(\umatrix + \vmatrix) + (-\frac{1}{2})\wmatrix
    \\
    \label{eq:19}
    && &= 5\begin{bmatrix}
    0.5\\3\\-1
    \end{bmatrix}
    +(-\frac{1}{2})\wmatrix
    \\
    \label{eq:20}
    && &= 5\begin{bmatrix}
    0.5\\3\\-1
    \end{bmatrix}
    + 5\begin{bmatrix}
    0.5\\3\\-1
    \end{bmatrix}
    \\
    \label{eq:21}
    \text{by 4a} && &= 5(\begin{bmatrix}
    0.5\\3\\-1
    \end{bmatrix} + \begin{bmatrix}
    0.5\\3\\-1
    \end{bmatrix})
    \\
    \label{eq:22}
    \text{by 4b} && &= 5(1 + 1)\begin{bmatrix}
    0.5\\3\\-1
    \end{bmatrix}
    \\
    \label{eq:23}
    && &= \begin{bmatrix}
    5 \\ 30 \\ -10
    \end{bmatrix}
\end{align}

\subsubsection*{ii) -20(\textbf{u} + \textbf{v}) + 2\textbf{w}}

\begin{align}
    \label{eq:24}
    \text{by 4a} && -20(\textbf{u} + \textbf{v}) + 2\textbf{w} &= -4[5(\textbf{u} + \textbf{v}) -\frac{1}{2}\textbf{w}]
    \\
    \label{eq:25}
    \eqref{eq:23} && &= -4\begin{bmatrix}
    5 \\ 30 \\ -10
    \end{bmatrix}
    \\
    \label{eq:26}
    && &= \begin{bmatrix}
    -20\\-120\\40
    \end{bmatrix}
\end{align}

\subsubsection*{iii) \textbf{E} - (\textbf{u} + \textbf{v})}

\begin{align}
    \label{eq:27}
    \eqref{eq:16} && \textbf{E} - (\textbf{u} + \textbf{v}) &= \Ematrix{} - (\umatrix + \vmatrix)
\end{align}
\par \hspace{15mm} Not defined. Matrices of different size (i.e. 3$\times$2 vs. 3$\times$1) cannot be added/subtracted.

\subsubsection*{iv) 10(\textbf{u} + \textbf{v}) + \textbf{w}}

\begin{align}
    \label{eq:28}
    \eqref{eq:16} && 10(\textbf{u} + \textbf{v}) + \textbf{w} &= 10(\textbf{u} + \textbf{v}) + \wmatrix
    \\
    \label{eq:29}
    && &= 10(\textbf{u} + \textbf{v}) + 10\begin{bmatrix}
    -0.5\\-3\\1
    \end{bmatrix}
    \\
    \label{eq:30}
    \text{by 4a} && &= 10(\textbf{u} + \textbf{v} + \begin{bmatrix}
    -0.5\\-3\\1
    \end{bmatrix})
    \\
    \label{eq:31}
    \eqref{eq:19} && &= 10(\begin{bmatrix}
    0.5\\0\\-3.0
    \end{bmatrix} + \begin{bmatrix}
    -0.5\\-3\\1
    \end{bmatrix})
    \\
    \label{eq:32}
    \text{by 3d} && &= (10)\textbf{0}
    \\
    \label{eq:33}
    && &= \textbf{0}
\end{align}

\newpage

\section*{7.2}
\setcounter{equation}{0}

\subsection*{7.2 [6]}
\setcounter{equation}{0}

\par If \textbf{U\textsubscript{1}}, \textbf{U\textsubscript{2}} are upper triangle and \textbf{L\textsubscript{1}}, \textbf{L\textsubscript{2}} are lower triangular, which of the following are triangular?

\def \Uonematrix {
\begin{bmatrix}
u & u & u\\
0 & u & u\\
0 & 0 & u
\end{bmatrix}
}
\def \Utwomatrix {
\begin{bmatrix}
v & v & v\\
0 & v & v\\
0 & 0 & v
\end{bmatrix}
}

\def \Lonematrix {
\begin{bmatrix}
l & 0 & 0\\
l & l & 0\\
l & l & l
\end{bmatrix}
}
\def \Ltwomatrix {
\begin{bmatrix}
m & 0 & 0\\
m & m & 0\\
m & m & m
\end{bmatrix}
}

Assuming the matrices are all the same size, let
\begin{align}
    \label{eq:1}
    \textbf{U\textsubscript{1}} = \Uonematrix{} &&
    \textbf{U\textsubscript{2}} = \Utwomatrix{} &&
    \textbf{L\textsubscript{1}} = \Lonematrix{} &&
    \textbf{L\textsubscript{2}} = \Ltwomatrix{}
\end{align}

\subsubsection*{i) \textbf{U\textsubscript{1}} + \textbf{U\textsubscript{2}} is triangular:} 

\begin{align}
    \label{eq:2}
    \eqref{eq:1} &&
    \textbf{U\textsubscript{1}} + \textbf{U\textsubscript{2}} &=
    \begin{bmatrix}
    u+v & u+v & u+v\\
    0 & u+v & u+v\\
    0 & 0 & u+v
    \end{bmatrix}
\end{align}

\subsubsection*{ii) \textbf{U\textsubscript{1}}\textbf{U\textsubscript{2}} is triangular:}

\begin{align}
    \label{eq:3}
    \eqref{eq:1} &&
    \textbf{U\textsubscript{1}}\textbf{U\textsubscript{2}} &=
    \begin{bmatrix}
    uv & 2uv & 3uv\\
    0 & uv & 2uv\\
    0 & 0 & uv
    \end{bmatrix}
\end{align}

\subsubsection*{iii) \textbf{U\textsubscript{1}\textsuperscript{2}} is triangular:}

\begin{align}
    \label{eq:4}
    \eqref{eq:1} &&
    \textbf{U\textsubscript{1}\textsuperscript{2}} &= 
    \begin{bmatrix}
    u\textsuperscript{2} & 2u\textsuperscript{2} & 3u\textsuperscript{2}\\
    0 & u\textsuperscript{2} & 2u\textsuperscript{2}\\
    0 & 0 & u\textsuperscript{2}
    \end{bmatrix}
\end{align}

\subsubsection*{iv) \textbf{U\textsubscript{1}} + \textbf{L\textsubscript{1}} is not triangular:}

\begin{align}
    \label{eq:5}
    \eqref{eq:1} &&
    \textbf{U\textsubscript{1}} + \textbf{L\textsubscript{1}} &=
    \begin{bmatrix}
    u + l & u & u\\
    l & u + l & u\\
    l & l & u + l
    \end{bmatrix}
\end{align}

\subsubsection*{v) \textbf{U\textsubscript{1}}\textbf{L\textsubscript{1}} is not triangular:}

\begin{align}
    \label{eq:6}
    \eqref{eq:1} &&
    \textbf{U\textsubscript{1}}\textbf{L\textsubscript{1}} &=
    \begin{bmatrix}
    3ul & 2ul & ul\\
    2ul & 2ul & ul\\
    ul & ul & ul
    \end{bmatrix}
\end{align}

\subsubsection*{vi) \textbf{L\textsubscript{1}} + \textbf{L\textsubscript{2}} is triangular:}

\begin{align}
    \label{eq:7}
    \eqref{eq:1} &&
    \textbf{L\textsubscript{1}} + \textbf{L\textsubscript{2}} &=
    \begin{bmatrix}
    l+m & 0 & 0\\
    l+m & l+m & 0\\
    l+m & l+m & l+m
    \end{bmatrix}
\end{align}

\subsection*{7.2 [12]}
\setcounter{equation}{0}

Showing all intermediate results, calculate the following expressions or give reasons why they are undefined:

\def \Amatrix {
\begin{bmatrix}
4 & -2 & 3\\
-2 & 1 & 6\\
1 & 2 & 2
\end{bmatrix}
}
\def \Bmatrix {
\begin{bmatrix}
1 & -3 & 0\\
-3 & 1 & 0\\
0 & 0 & -2
\end{bmatrix}
}

\begin{align}
\label{eq:1}
\textbf{A} = \Amatrix && \textbf{B} = \Bmatrix
\end{align}

\subsubsection*{i) \textbf{AA}\textsuperscript{T}}

\begin{align}
    \label{eq:2}
    \eqref{eq:1} &&
    \textbf{A}\textsuperscript{T} &= 
    \begin{bmatrix}
    4 & -2 & 1\\
    -2 & 1 & 2\\
    3 & 6 & 2
    \end{bmatrix}
    \\
    \label{eq:3}
    \eqref{eq:1}, \eqref{eq:2} &&
    \textbf{AA}\textsuperscript{T} &=
    \Amatrix \begin{bmatrix}
    4 & -2 & 1\\
    -2 & 1 & 2\\
    3 & 6 & 2
    \end{bmatrix}
    \\
    \label{eq:4}
    && &= \begin{bmatrix}
    16+4+9 & -8-2+18 & 4-4+6\\
    -8-2+18 & 4+1+36 & -2+2+12\\
    4-4+6 & -2+2+12 & 1+4+4
    \end{bmatrix}
    \\
    \label{eq:5}
    && &= \begin{bmatrix}
    29 & 8 & 6\\
    8 & 41 & 12\\
    6 & 12 & 9
    \end{bmatrix}
\end{align}

\subsubsection*{ii) \textbf{A}\textsuperscript{2}}

\begin{align}
    \label{eq:6}
    \eqref{eq:1} &&
    \textbf{A}\textsuperscript{2} &= \Amatrix \Amatrix
    \\
    \label{eq:7}
    && &= \begin{bmatrix}
    16+4+3 & -8-2+6 & 12-12+6 \\
    -8-2+6 & 4+1+12 & -6+6+12 \\
    4-4+2 & -2+2+4 & 3+12+4
    \end{bmatrix}
    \\
    \label{eq:8}
    && &= \begin{bmatrix}
    23 & -4 & 6\\
    -4 & 17 & 12\\
    2 & 4 & 19
    \end{bmatrix}
\end{align}

\subsubsection*{iii) \textbf{BB}\textsuperscript{T}}

\begin{align}
    \label{eq:9}
    \eqref{eq:1} &&
    \textbf{B}\textsuperscript{T} &= \Bmatrix
    \\
    \label{eq:10}
    \eqref{eq:1} &&
    &= \textbf{B}
    \\
    \label{eq:11}
    \eqref{eq:10} &&
    \textbf{BB}\textsuperscript{T} &= \textbf{B}\textsuperscript{2}
    \\
    \label{eq:12}
    \eqref{eq:1} &&
    &= \Bmatrix \Bmatrix
    \\
    \label{eq:13}
    && &= \begin{bmatrix}
    1+9+0 & -3-3+0 & 0+0+0\\
    -3-3+0 & 9+1+0 & 0+0+0\\
    0+0+0 & 0+0+0 & 0+0+4
    \end{bmatrix}
    \\
    \label{eq:14}
    && &= \begin{bmatrix}
    10 & -6 & 0\\
    -6 & 10 & 0\\
    0 & 0 & 4
    \end{bmatrix}
\end{align}

\subsubsection*{iv) \textbf{B}\textsuperscript{2}}

\begin{align}
    \label{eq:15}
    \eqref{eq:11}, \eqref{eq:14} &&
    \textbf{B}\textsuperscript{2} &= \begin{bmatrix}
    10 & -6 & 0\\
    -6 & 10 & 0\\
    0 & 0 & 4
    \end{bmatrix}
\end{align}

\subsection*{7.2 [16]}
\setcounter{equation}{0}

Showing all intermediate results, calculate the following expressions or give reasons why they are undefined:

\def \Cmatrix {
\begin{bmatrix}
0 & 1\\
3 & 2\\
-2 & 0
\end{bmatrix}
}
\def \bbmatrix {
\begin{bmatrix}
3\\
1\\
-1
\end{bmatrix}
}

\begin{align}
    \label{eq:1}
    \textbf{B} = \Bmatrix && \textbf{C} = \Cmatrix && \textbf{b} = \bbmatrix
\end{align}

\subsubsection*{i) \textbf{BC}}

\begin{align}
    \label{eq:2}
    \eqref{eq:1} &&
    \textbf{BC} &= \Bmatrix \Cmatrix
    \\
    \label{eq:3}
    && &= \begin{bmatrix}
    0-9+0 & 1-6+0\\
    0+3+0 & -3+2+0\\
    0+0+4 & 0+0+0
    \end{bmatrix}
    \\
    \label{eq:4}
    && &= \begin{bmatrix}
    -9 & -5\\
    3 & -1\\
    4 & 0
    \end{bmatrix}
\end{align}

\subsubsection*{ii) \textbf{BC}\textsuperscript{T}}

\begin{align}
    \label{eq:5}
    \eqref{eq:1} &&
    \textbf{C}\textsuperscript{T} &= \begin{bmatrix}
    0 & 3 & -2\\
    1 & 2 & 0
    \end{bmatrix}
    \\
    && \textbf{BC}\textsuperscript{T} & \text{\indent DNE}
\end{align}

Multiplication of a 3$\times$3 matrix (\textbf{B}) by a 2$\times$3 matrix (\textbf{C}\textsuperscript{T}) is undefined.  The number of columns in \textbf{B} (in this case, 3) has to equal the number of rows in \textbf{C}\textsuperscript{T} (in this case, 2) for matrix multiplication to be defined.

\subsubsection*{iii) \textbf{Bb}}

\begin{align}
    \label{eq:7}
    \eqref{eq:1} &&
    \textbf{Bb} &= \Bmatrix \bbmatrix
    \\
    \label{eq:8}
    && &= \begin{bmatrix}
    3-3+0\\
    -9+1+0\\
    0+0+2
    \end{bmatrix}
    \\
    \label{eq:9}
    && &= \begin{bmatrix}
    0\\
    -8\\
    2
    \end{bmatrix}
\end{align}

\subsubsection*{iv) \textbf{b}\textsuperscript{T}\textbf{B}}

\begin{align}
    \label{eq:10}
    \eqref{eq:1} &&
    \textbf{b}\textsuperscript{T} &= \begin{bmatrix}
    3 & 1 & -1
    \end{bmatrix}
    \\
    \label{eq:11}
    \eqref{eq:1}, \eqref{eq:10} &&
    \textbf{b}\textsuperscript{T}\textbf{B} &= \begin{bmatrix}
    3 & 1 & -1
    \end{bmatrix} \Bmatrix
    \\
    \label{eq:12}
    && &= \begin{bmatrix}
    3-3+0 & -9+1+0 & 0+0+2
    \end{bmatrix}
    \\
    \label{eq:13}
    && &= \begin{bmatrix}
    0 & -8 & 2
    \end{bmatrix}
\end{align}

\par  $ $

\subsection*{7.2 [18]}
\setcounter{equation}{0}

Showing all intermediate results, calculate the following expressions or give reasons why they are undefined:

\def \aamatrix {
\begin{bmatrix}
1 & -2 & 0
\end{bmatrix}
}

\begin{align}
    \label{eq:1}
    \textbf{A} = \Amatrix && \textbf{B} = \Bmatrix && \textbf{a} = \aamatrix && \textbf{b} = \bbmatrix
\end{align}

\subsubsection*{i) \textbf{ab}}

\begin{align}
    \label{eq:2}
    \eqref{eq:1} &&
    \textbf{ab} &= \aamatrix \bbmatrix
    \\
    \label{eq:3}
    && &= \begin{bmatrix}
    3-2+0
    \end{bmatrix}
    \\
    \label{eq:4}
    && &= \begin{bmatrix}
    1
    \end{bmatrix}
\end{align}

\subsubsection*{ii) \textbf{ba}}

\begin{align}
    \label{eq:5}
    \eqref{eq:1} &&
    \textbf{ba} &= \bbmatrix \aamatrix
    \\
    \label{eq:6}
    && &= \begin{bmatrix}
    3 & -6 & 0\\
    1 & -2 & 0\\
    -1 & 2 & 0
    \end{bmatrix}
\end{align}

\subsubsection*{iii) \textbf{aA}}

\begin{align}
    \label{eq:7}
    \eqref{eq:1} &&
    \textbf{aA} &= \aamatrix \Amatrix
    \\
    \label{eq:8}
    && &= \begin{bmatrix}
    4+4+0 & -2-2+0 & 3-12+0
    \end{bmatrix}
    \\
    \label{eq:9}
    && &= \begin{bmatrix}
    8 & -4 & -9
    \end{bmatrix}
\end{align}

\subsubsection*{iv) \textbf{Bb}}

\begin{align}
    \label{eq:10}
    \text{from 7.2 [16] (iii), eq. (9)} && 
    \textbf{Bb} &= \begin{bmatrix}
    0\\
    -8\\
    2
    \end{bmatrix}
\end{align}

\newpage

\section*{7.3}
\setcounter{equation}{0}

\subsection*{7.3 [2]}
\setcounter{equation}{0}

Solve the linear system given explicitly or by its augmented matrix.  Show details.

\begin{align}
    \label{eq:1}
    && & \begin{bmatrix}
    3.0 & -0.5 & 0.6\\
    1.5 & 4.5 & 6.0
    \end{bmatrix}
    \\
    \label{eq:2}
    \textbf{R\textsubscript{2}} = \textbf{R\textsubscript{2}} + (-\frac{1}{2})\textbf{R\textsubscript{1}}  \longrightarrow && &= \begin{bmatrix}
    3.0 & -0.5 & 0.6\\
    1.5-1.5 & 4.5+0.25 & 6.0-0.3
    \end{bmatrix}
    \\
    \label{eq:3}
    && &= \begin{bmatrix}
    3.0 & -0.5 & 0.6\\
    0 & 4.75 & 5.7
    \end{bmatrix}
    \\
    \label{eq:4}
    \textbf{R\textsubscript{2}} = \frac{1}{4.75}\textbf{R\textsubscript{2}} 
    \longrightarrow && &= \begin{bmatrix}
    3.0 & -0.5 & 0.6\\
    0 & 1 & 1.2
    \end{bmatrix}
\end{align}
    
\text{This represents the following linear system:}

\begin{align}
    \label{eq:5}
    \eqref{eq:4} && 
    3.0x_1 + 0.5x_2 &= 0.6
    \\
    \label{eq:6}
    && x_2 &= 1.2
    \\
    \label{eq:7}
    \eqref{eq:6} \rightarrow \eqref{eq:5} && 
    3.0x_1 + 0.5(1.2) &= 0.6
    \\
    \label{eq:8}
    && x_1 &= 0.4
\end{align}

From lines \eqref{eq:8}, \eqref{eq:6}:
\begin{empheq}[box=\fbox]{align*}
    x_1 &= 0.4 \\
    x_2 &= 1.2
\end{empheq}

\subsection*{7.3 [8]}
\setcounter{equation}{0}

Solve the linear system given explicitly or by its augmented matrix. Show details.

\begin{align}
    \label{eq:1}
    4y + 3z &= 8\\
    \label{eq:2}
    2x - z &= 2\\
    \label{eq:3}
    3x + 2y &= 5
\end{align}

This can be represented by the matrix

\begin{align}
    \label{eq:4}
    && \begin{bmatrix}
    0 & 4 & 3 & 8\\
    2 & 0 & -1 & 2\\
    3 & 2 & 0 & 5
    \end{bmatrix} &= \begin{bmatrix}
    3 & 2 & 0 & 5\\
    2 & 0 & -1 & 2\\
    0 & 4 & 3 & 8
    \end{bmatrix}
    \\
    \label{eq:5}
    \textbf{R\textsubscript{2}} = \textbf{R\textsubscript{2}} + (-\frac{2}{3})\textbf{R\textsubscript{1}} \longrightarrow && &= \begin{bmatrix}
    3 & 2 & 0 & 5\\
    2-2 & 0-\frac{4}{3} & -1 & 2-\frac{10}{3}\\
    0 & 4 & 3 & 8
    \end{bmatrix}
    \\
    \label{eq:6}
    && &= \begin{bmatrix}
    3 & 2 & 0 & 5\\
    0 & -\frac{4}{3} & -1 & -\frac{4}{3}\\
    0 & 4 & 3 & 8
    \end{bmatrix}
    \\
    \label{eq:7}
    \textbf{R\textsubscript{3}} = \textbf{R\textsubscript{3}} + (3)\textbf{R\textsubscript{1}} \longrightarrow && &= \begin{bmatrix}
    3 & 2 & 0 & 5\\
    0 & -\frac{4}{3} & -1 & -\frac{4}{3}\\
    0 & 4-4 & 3-3 & 8-4
    \end{bmatrix}
    \\
    \label{eq:8}
    && &= \begin{bmatrix}
    3 & 2 & 0 & 5\\
    0 & -\frac{4}{3} & -1 & -\frac{4}{3}\\
    0 & 0 & 0 & 4
    \end{bmatrix}
\end{align}

\textbf{R\textsubscript{3}} represents the equation $0=4$, which is not valid.  Hence, \textbf{there is no solution} to the linear system.

\end{document}
