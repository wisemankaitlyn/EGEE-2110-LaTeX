\documentclass{article}
\usepackage[utf8]{inputenc}
\usepackage[margin=1in,headheight=24pt]{geometry}
\usepackage{amsmath}
\usepackage{lastpage}
\usepackage{fancyhdr}
\pagestyle{fancy}
\allowdisplaybreaks

\fancyhead{}
\fancyhead[L]{EGEE-2110 \\ Engineering Analysis}
\fancyhead[C]{Week 5 Homework}
\fancyhead[R]{02/19/2021 \\ Kaitlyn Wiseman}
\renewcommand{\footrulewidth}{0.4pt}
\fancyfoot{}
\fancyfoot[R]{\thepage\ of \pageref{LastPage}}


\begin{document}

{\large \noindent Problem Set:}

\par 8.3 [1, 2, 3, 4, 5]
\par 8.4 [1, 2, 9, 10]
\par 9.1 [2, 6, 22]
\par 9.2 [2, 4, 6, 8, 22, 36]
\vspace{5mm}

\noindent \hrulefill

\section*{8.3}
\setcounter{equation}{0}

\subsection*{8.3 [1]}
\setcounter{equation}{0}

Is the following matrix symmetric, skew-symmetric, or orthogonal?  Find the spectrum of the matrix, thereby illustrating Theorems 1 and 5.

\begin{align}
    \label{eq:1}
    && \textbf{A} &= \begin{bmatrix} 
    0.8 & 0.6 \\
    -0.6 & 0.8
    \end{bmatrix}
    \\
    \label{eq:2}
    && \textbf{A\textsuperscript{T}} &= \begin{bmatrix}
    0.8 & -0.6\\
    0.6 & 0.8
    \end{bmatrix}
    \\
    \label{eq:3}
    \eqref{eq:1}, \eqref{eq:2} && \textbf{AA\textsuperscript{T}} &= \begin{bmatrix}
    0.8 & 0.6 \\
    -0.6 & 0.8
    \end{bmatrix} \begin{bmatrix}
    0.8 & -0.6 \\
    0.6 & 0.8
    \end{bmatrix}
    \\
    \label{eq:4}
    && &= \begin{bmatrix}
    1 & 0\\
    0 & 1
    \end{bmatrix}
    \\
    \label{eq:5}
    && &= \textbf{I}
\end{align}

Hence, \textbf{A} is an \textbf{orthogonal} matrix.

\begin{align}
    \label{eq:6}
    \begin{vmatrix}
    \textbf{A} - \lambda\textbf{I}
    \end{vmatrix} = 0 \rightarrow && \begin{vmatrix}
    (0.8-\lambda) & 0.6\\
    -0.6 & (0.8 - \lambda)
    \end{vmatrix} &= 0
    \\
    \label{eq:7}
    && \lambda^2 -0.6\lambda +1 &= 0
    \\
    \label{eq:8}
    && \lambda_1 &= 0.8 + 0.6j
    \\
    \label{eq:9}
    && \lambda_2 &= 0.8 - 0.6j
    \\
    \label{eq:10}
    && \begin{Vmatrix} \lambda_1 \end{Vmatrix} &= 1
    \\
    \label{eq:11}
    && \begin{Vmatrix} \lambda_2 \end{Vmatrix} &= 1
\end{align}

The absolute magnitudes of each eigenvalue of this orthogonal matrix are 1, illustrating Theorem 5.

\subsection*{8.3 [2]}
\setcounter{equation}{0}

Is the following matrix symmetric, skew-symmetric, or orthogonal?  Find the spectrum of the matrix, thereby illustrating Theorems 1 and 5.

\begin{align}
    \label{eq:1}
    && \textbf{A} &= \begin{bmatrix}
        a & b\\
        -b & a
    \end{bmatrix}
    \\
    \label{eq:2}
    && \textbf{A\textsuperscript{T}} &= \begin{bmatrix}
    a & -b\\
    b & a
    \end{bmatrix}
\end{align}

\textbf{A} is symmetric iff $b = 0$, skew-symmetric iff $a = 0$.

\begin{align}
    \label{eq:3}
    && \textbf{AA\textsuperscript{T}} &= \begin{bmatrix} 
    a & b\\
    -b & a
    \end{bmatrix} \begin{bmatrix} 
    a & -b\\
    b & a
    \end{bmatrix}
    \\
    \label{eq:4}
    && &= \begin{bmatrix}
    a^2+b^2 & 0\\
    0 & a^2+b^2
    \end{bmatrix}
    \\
    \label{eq:5}
    && &= \textbf{I}
\end{align}

if $a^2+b^2=1$, making \textbf{A} orthogonal.

Find eigenvalues:

\begin{align}
    \label{eq:6}
    \begin{vmatrix}
    \textbf{A}-\lambda\textbf{I}
    \end{vmatrix} = 0 \rightarrow && \begin{vmatrix}
    (a-\lambda) & b\\
    -b & (a-\lambda)
    \end{vmatrix} &= 0
    \\
    \label{eq:7}
    && \lambda ^2 -2a \lambda +a^2 + b^2 &= 0
\end{align}

\subsection*{8.3 [3]}
\subsection*{8.3 [4]}
\subsection*{8.3 [5]}

\newpage

\section*{8.4}
\setcounter{equation}{0}

\subsection*{8.4 [1]}
\subsection*{8.4 [2]}
\subsection*{8.4 [9]}
\subsection*{8.4 [10]}

\newpage

\section*{9.1}
\setcounter{equation}{0}

\subsection*{9.1 [2]}
\subsection*{9.1 [6]}
\subsection*{9.1 [22]}

\newpage

\section*{9.2}
\setcounter{equation}{0}

\subsection*{9.2 [2]}

\subsubsection*{i) lorem :)}

\begin{align}
    \label{eq:label}
    have fun math time && &=
    \\
    \label{eq:label2}
    \eqref{eq:label} && &=
\end{align}
\subsubsection*{ii) lorem ii :)}

\subsection*{9.2 [4]}

\subsubsection*{i) ipsum :)}

\subsection*{9.2 [6]}
\subsection*{9.2 [8]}
\subsection*{9.2 [22]}
\subsection*{9.2 [36]}

\end{document}